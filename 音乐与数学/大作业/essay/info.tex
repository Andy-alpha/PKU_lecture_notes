\title{论鼓的振动模态解和定音鼓的发声机理} % Article title


\author[1]{袁云鹏, 李兆彬}
\author[2]{蒲天舒, 肖钰洋} % Authors
\affil[1]{\textit{北京大学信息科学技术学院}}
\affil[2]{\textit{北京大学物理学院}}
%\affiliation{
%	\quad
%	\textsuperscript{1}\textit{北京大学信息科学技术学院}
%	\qquad
%	\textsuperscript{2}\textit{北京大学物理学院}
%	\qquad
%	*\textbf{通讯作者}: yuan_yunpeng@stu.pku.edu.cn
%} % Author affiliation

%\Abstract{
%	\phantom{田田}玄学又称新道家,是对《老子》、《庄子》和《周易》的研究和解说,产生于魏晋。玄学是中国魏晋时期到宋朝中叶之间出现的一种崇尚老庄的思潮。也可以说是道家之学的一种新的表现方式,故又有新道家之称。其思潮持续时间自汉末起至宋朝中叶结束。玄学是魏晋时期取代两汉经学思潮的思想主流。 玄学即“玄远之学”,它以“祖述老庄”立论,把《老子》、《庄子》、《周易》称作“三玄”。道家玄学也是除了儒学外唯一被定为官学的学问。
%}


%\Keywords{\phantom{田田}玄学\quad玄学\quad玄学} % 如不需要关键词可直接删去花括号中内容

